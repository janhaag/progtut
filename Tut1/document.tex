\batchmode
\documentclass{beamer}
\usepackage[utf8]{inputenc}
\usepackage[ngerman]{babel}

\usetheme[deutsch]{KIT}
\author{Jan Haag (jan.haag@student.kit.edu)}
\title{Programmieren Tutorium 1 -- Klassen und Objekte}
\institute{Institut f\"{u}r Theoretische Informatik}
\TitleImage[scale=0.225]{frontpic.jpg}

\begin{document}
\begin{frame}
\maketitle
\end{frame}

\begin{frame}
\frametitle{Inhalt}
\tableofcontents
\end{frame}

\section{Organisatorisches}
\begin{frame}[fragile]
\frametitle{Wichtige Daten}
\begin{description}
\item[Folienarchiv] \verb|https://github.com/janhaag/progtut2011|
\item[Fragen]
\begin{itemize}
\item Praktomat-Forum
\item Mir eine Mail schreiben (jan.haag@student.kit.edu)
\end{itemize}
\end{description}
\end{frame}

\section{Java -- Installation und Verwendung}
\begin{frame}[fragile]
\frametitle{Java -- Installation}
\begin{description}
\item[JDK Installation]
\begin{description}
\item[Windows] Download des Installers von der Java Download-Seite
\item[Linux] Die meisten Distributionen haben das OpenJDK oder das Oracle JDK in ihren Repositories
\item[Mac OS X] Das JDK wird zusammen mit den XCode Developer Tools von der Installations-DVD installiert.
\end{description}
\end{description}
\end{frame}

\begin{frame}[fragile]
\frametitle{Java -- Installation}
\begin{description}
\item[Editoren]
\begin{description}
\item[Windows] Notepad++, Vim
\item[Linux] Vim, Emacs, Kate, gEdit, ...
\item[Mac OS X] MacVim, XCode, TextMate, ...
\end{description}
\item[Java Docs] Download von der Java Download-Seite
\item[Java Downloads] \verb|http://www.oracle.com/technetwork/java/|$\hookleftarrow$\\
\verb|javase/downloads/index.html|
\end{description}
\end{frame}

\begin{frame}
\frametitle{Funktion des JDK}
\end{frame}


\section{Konventionen}
\begin{frame}
    \frametitle{Konventionen}
    \begin{itemize}
        \item Helfen, code verst\"{a}ndlich zu halten
        \item Helfen programmen, die Code analysieren
            \pause
        \item Sorgen daf\"{u}r, dass andere (auch ihr selbst, in 8 wochen!) nachvollziehen k\"{o}nnen, was der code tut.
            \pause
        \item \emph{Nichteinhaltung f\"{u}hrt zu Punktabzug!}
    \end{itemize}
\end{frame}

\subsection{Namen}
\begin{frame}[fragile]
    \frametitle{Namenskonventionen}
    \begin{itemize}
        \item Namen in CamelCase
        \item Klassennamen gross (\verb|class MyClass { XXX }|)
        \item Variablennamen klein, Namen i. d. R. Substantive (\verb|int importantIntVariable;|)
        \item Laufvariablen mit einem buchstaben, Namen i. d. R anfangsbuchstabe des
            typen oder darauf folgende buchstaben (\verb|for(int i = 0; i < 10; i++) { XXX }|)
        \item Methodennamen klein, Namen i. d. R. Verben (\verb|int doSomething(int foo) { XXX }|)
        \item Methoden zum Lesen oder Schreiben von Attributen hei\ss{}en getAttr bzw. setAttr
        \item Konstanten in gro\ss{}buchstaben, w\"{o}rter durch \verb|_|
            getrennt (\verb|public static final int IMPORTANT_CONSTANT = 1;|)
    \end{itemize}
\end{frame}

\begin{frame}
    \frametitle{Dateien und Inhalte}
    \begin{itemize}
        \item Jede Klasse in eine eigene Datei
        \item Dateiname: Klassenname.java
    \end{itemize}
\end{frame}

\section{Java -- Funktion}
\begin{frame}[fragile]
\frametitle{Aufbau einer Klasse}
\begin{verbatim}
public class Example {
    Typ attribut1;
    int zahl = 1;
}
\end{verbatim}
\end{frame}

\subsection{Aufgaben}
\begin{frame}
\frametitle{Modellierung eines Autos}
Was macht ein Auto aus? Modelliere es durch passende Klassen und Objekte.
\end{frame}

\begin{frame}[fragile]
\frametitle{Theorie\ldots}
Werte folgende Ausdr\"{u}cke aus.
\begin{verbatim}
boolean a = true;
boolean b = false;
boolean c = true;
boolean d;
d = !a;
d = a && b;
d = !a || !c;
d = (a && b) || !c;
\end{verbatim}
\end{frame}

\begin{frame}[fragile]
\frametitle{Theorie\ldots}
Werte folgende Ausdr\"{u}cke aus.
\begin{verbatim}
int i = 2;
double d = 3.14;

int result = i + 12;
double result = i / 5;
int result = i / 6;
int result = 7 % i;
int result = 7 / i;
double result = i / d;
double result = 2 / 3;
double result = 2 / 3.0;
\end{verbatim}
\end{frame}

\begin{frame}
    \frametitle{\"Ubungsaufgabe}
    Modelliere die Lagerhaltung in einem Autoteile-Handel.
\end{frame}

\begin{frame}
\frametitle{Ende}
\begin{center}
\includegraphics[scale=.28]{good_code.png}
\end{center}
\end{frame}
\end{document}
