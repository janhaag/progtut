\batchmode
\documentclass{beamer}
\usepackage[utf8]{inputenc}
\usepackage[ngerman]{babel}
\usepackage{amsmath}

\usetheme[deutsch]{KIT}
\author{Jan Haag (jan.haag@student.kit.edu)}
\title{Programmieren Tutorium 5 -- Widerholung}
\institute{Institut f\"ur Theoretische Informatik}
\TitleImage[scale=0.225]{frontpic.jpg}

\begin{document}
\begin{frame}
\maketitle
\end{frame}

\begin{frame}
\frametitle{Inhalt}
\tableofcontents
\end{frame}

\section{Sichtbarkeiten}
\begin{frame}[fragile]
\frametitle{Sichtbarkeiten}
\begin{tabular}{l|l}
Sichtbarkeit & Wirkung\\
\hline
\verb|public| & F\"{u}r alle sichtbar\\
 & F\"{u}r alle innerhalb des gleichen package sichtbar\\
\verb|protected| & Nur f\"{u}r \glqq{}verwandte\grqq{} sichtbar\\
\verb|private| & Nur innerhalb der Klasse sichtbar\\
\end{tabular}
\end{frame}

\section{Aufgabe}
\begin{frame}[fragile]
\frametitle{Tipps}
\begin{verbatim}
img.getMinY();
img.getHeight();
new Color(img.getRGB(x, y));
img.setRGB(x, y, color.getRGB());
\end{verbatim}
\end{frame}

\begin{frame}[fragile]
\frametitle{Aufgabe}
\begin{verbatim}
for each y from top to bottom
   for each x from left to right
      oldpixel := pixel[x][y]
      newpixel := find_closest_palette_color(oldpixel)
      pixel[x][y] := newpixel
      quant_error := oldpixel - newpixel
      pixel[x+1][y] := pixel[x+1][y] + 7/16 * quant_error
      pixel[x-1][y+1] := pixel[x-1][y+1] + 3/16 * quant_error
      pixel[x][y+1] := pixel[x][y+1] + 5/16 * quant_error
      pixel[x+1][y+1] := pixel[x+1][y+1] + 1/16 * quant_error
\end{verbatim}
\end{frame}

\begin{frame}
\frametitle{Ende}
\includegraphics[scale=0.4]{tornadoguard.png}
\end{frame}
\end{document}
