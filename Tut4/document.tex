\batchmode
\documentclass{beamer}
\usepackage[utf8]{inputenc}
\usepackage[ngerman]{babel}
\usepackage{amsmath}

\usetheme[deutsch]{KIT}
\author{Jan Haag (jan.haag@student.kit.edu)}
\title{Programmieren Tutorium 4 -- Arrays}
\institute{Institut f\"ur Theoretische Informatik}
\TitleImage[scale=0.225]{frontpic.jpg}

\begin{document}
\begin{frame}
\maketitle
\end{frame}

\begin{frame}
\frametitle{Inhalt}
\tableofcontents
\end{frame}

%\section{Sichtbarkeiten}
%\begin{frame}[fragile]
%\frametitle{Sichtbarkeiten}
%\begin{tabular}{l|l}
%Sichtbarkeit & Wirkung\\
%\hline
%\verb|public| & F\"{u}r alle sichtbar\\
% & F\"{u}r alle innerhalb des gleichen package sichtbar\\
%\verb|protected| & Nur f\"{u}r \glqq{}verwandte\grqq{} sichtbar\\
%\verb|private| & Nur innerhalb der Klasse sichtbar\\
%\end{tabular}
%\end{frame}

\section{Korrektur 1. \"Ubungsblatt}
\begin{frame}[fragile]
    \frametitle{Korrektur 1. \"Ubungsblatt}
    \begin{itemize}
        \item \verb|1.0 - 0.9 == 0.1| $\Rightarrow$ \verb|false|
        \item Unsinnige oder falsche Kommentare sind schlimmer als keine!
        \item Variablen sollten nur deklariert werden, wenn man sie
            auch Verwendet
        \item Der Praktomat mag keine Umlaute und Verzeichnisstrukturen
    \end{itemize}
\end{frame}

\section{Arrays}
\begin{frame}
\frametitle{Arrays -- Motivation}
\begin{itemize}
\item Speichern eine beliebige Menge an Daten (auch andere Arrays)
\item Sehr schnelles Lesen und Schreiben aller Elemente
\item Platzsparend
\end{itemize}
\pause
aber: Werteanzahl ist nach dem erstellen nicht mehr \"{a}nderbar!
\end{frame}

\begin{frame}[fragile]
\frametitle{Arrays}
\begin{verbatim}
public class Demo {
    private int[] demo;

    public Demo(int[] d) {
        demo = d;
    }

    public int getElem(int pos) {
        return demo[pos];
    }
\end{verbatim}
\end{frame}

\begin{frame}[fragile]
\frametitle{Arrays}
\begin{verbatim}
    public String toString() {
        String result = "";
        for (int i:demo) {
            result += i + " ";
        }
        return result;
    }
\end{verbatim}
\end{frame}

\begin{frame}[fragile]
\frametitle{Arrays}
\begin{verbatim}
    public static void main(String... args) {
        int[] is = new int[5];
        for (int i = 0; i < is.length; i++) {
            is[i] = i + 1;
        Demo d = new Demo(is);
        System.out.println(d.getElem(1));
        System.out.println(d);
    }
}
\end{verbatim}
\end{frame}

\section{Aufgaben}
\begin{frame}
\frametitle{Aufgabe}
$$(AB)_{i,j}=\sum^{p}_{k=1}A_{ik}B_{kj}$$
$n,p,m\in\mathbb{N}$; $A\in{}\mathbb{R}^{m\times{}p}$;
$B\in\mathbb{R}^{p\times{}n}$
\end{frame}

\begin{frame}
\frametitle{Ende}
\includegraphics[scale=0.4]{unpickable-1.png}
\end{frame}
\end{document}
